\documentclass[a4paper, 11pt, twoside]{article}
\usepackage[a4paper, left=2cm, right=2cm, top=2.5cm, bottom=2.5cm]{geometry}
\usepackage[utf8]{inputenc}
\usepackage{fancyhdr}
\usepackage{url}
\usepackage{microtype}
\usepackage[backend=biber, style=authoryear]{biblatex}

\author{Your Name}
\date{\today}
\title{Methodology, Justice, and Open Science in Energy Research}

\addbibresource{references.bib}

\begin{document}

\maketitle

\section*{Introduction}
The increasing complexity of energy systems demands a parallel sophistication in how we conduct research. The dual responsibility of research—to produce reliable knowledge and to do so in an ethically robust manner—is nowhere more apparent \parencite{Wieten2017}. This reflection provides a descriptive analysis of the deep connection between research methodology, ethical dimensions, and the role of open science in the energy sector. It argues that methodological choices are not merely technical decisions but are foundational ethical choices that define the moral landscape of an inquiry. Using the tenets of energy justice—distributive, procedural, and recognition—as an analytical lens \parencite{Sovacool2015}, this assignment maps how three families of research methodologies differently reveal these ethical terrains. Further, it proposes that specific, practical open science solutions can help researchers navigate these complexities, fostering research that is not only scientifically sound but also more transparent, just, and accountable.

\section*{Quantitative Methodologies, Distributive Justice, and Open Data}
Quantitative methodologies, which are composed of methods such as Cost-Benefit Analysis (CBA), provide powerful frameworks for standardization and are primarily oriented towards the ethical terrain of \textit{distributive justice}. They aim to create objective, numerical representations that allow for the efficient allocation of costs and benefits. In the context of large-scale energy projects, such as the Colectora transmission line in La Guajira, Colombia, CBAs are often used to justify developments based on national-level benefits like grid stability and carbon reduction. However, this focus on an aggregated, economic metric can obscure other profound injustices. The CBA framework, for example, struggles to value non-monetary assets like ancestral lands or cultural heritage, leading to deep ethical conflicts rooted in a failure of \textit{recognition justice} \parencite{VegaAraujo2022}.

The key ethical risk of these approaches—their ``black box'' nature—can be directly mitigated by the principles of open science. By embracing \textbf{open data}, researchers can make the underlying assumptions, datasets, and models of a CBA publicly available. This practice enhances transparency and contestability, allowing affected stakeholders to scrutinize inputs, challenge valuations, and even propose alternative models. While this does not remove the inherent limitations of CBA, it transforms an opaque, top-down decree into a more transparent and debatable tool, thereby enhancing \textit{procedural justice} by opening the process to wider review \parencite{AlonsoPedrero2025}.

\section*{Qualitative Methodologies, Recognition Justice, and Open Access}
In contrast, qualitative methodologies, which employ methods such as ethnography and semi-structured interviews, excel at exploring the terrain of \textit{recognition justice}. By focusing on lived experience, meaning, and context, they can uncover the diverse values, cultures, and perceptions of fairness that quantitative approaches often miss. The study by \textcite{VegaAraujo2022} is itself a prime example; using qualitative methods, the researchers uncovered a critical conflict between state-recognized and community-recognized leaders, an issue of legitimacy that a quantitative survey would likely have missed. This demonstrates how the choice of methodology determines whether crucial social nuances are ``seen'' by the research.

The open science solution in this context is twofold. First, \textbf{open access publishing} ensures that the research findings, which often give voice to marginalized communities, are not locked behind paywalls and are accessible to the communities themselves. Second, embracing \textbf{FAIR (Findable, Accessible, Interoperable, Reusable) data principles} provides a robust framework for sharing qualitative data ethically. This involves creating careful data management plans that protect participant anonymity and privacy while ensuring that the narratives and knowledge can be appropriately archived and shared, giving communities greater control over their own stories.

\section*{Participatory Methodologies, Procedural Justice, and Open-Source Hardware}
Participatory methodologies, such as Participatory Action Research (PAR), are explicitly designed to embody \textit{procedural justice}. They seek to break down the traditional researcher-subject hierarchy by engaging communities as active co-producers of knowledge. However, they often face a practical challenge in ensuring that participation is genuine and not merely superficial. Studies of opposition to wind farms in Australia, for instance, show that resistance is often driven not by the infrastructure itself, but by a perceived lack of procedural fairness in a top-down planning process \parencite{Gross2007}.

A powerful open science solution to this challenge lies in \textbf{Open-Source Appropriate Technology (OSAT)}. OSAT involves making the designs for technologies—from solar panels to water pumps—freely available so that communities can adapt, build, and maintain them \parencite{Pearce2019}. This practice provides tangible tools for community co-design and ownership, transforming participation from a consultative exercise into a process of genuine empowerment and co-production \parencite{Rana2023}. By providing open-source tools, researchers can facilitate a more profound form of procedural justice where communities become true partners in shaping their own technological futures.

\section*{Conclusion}
The choice of research methodology in the energy field is a critical ethical decision. A CBA, an ethnographic study, and a participatory workshop are not interchangeable tools; they are distinct frameworks that structure what is valued, who is heard, and how justice is defined. This reflection has provided a descriptive map showing how quantitative, qualitative, and participatory methodologies align with distributive, recognition, and procedural justice, respectively. More importantly, it has shown that open science is not an abstract ideal, but a set of practical tools—open data, open access, and open-source hardware—that can help researchers navigate the inherent ethical trade-offs. An awareness of this three-part relationship is essential for conducting responsible and impactful energy research that is both scientifically rigorous and ethically sound.

\printbibliography

\end{document}
