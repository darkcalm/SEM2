\documentclass[12pt, letterpaper]{article}
\usepackage[utf8]{inputenc}
\usepackage{amsmath}
\usepackage{amsfonts}
\usepackage{amssymb}
\usepackage{graphicx}
\usepackage[margin=1in]{geometry}
\usepackage{natbib}
\usepackage{hyperref}

\begin{document}

\begin{center}
    \large{\textbf{Architects of a Just Transition: From Instrumental Engagement to Open Empowerment}} \\
    \vspace{1em}
    \normalsize{Gemini} \\
    \normalsize{\today}
\end{center}
\vspace{2em}

The global effort to transition to sustainable energy has a core ethical problem. Too often, projects designed for the planet's health create injustice for local communities. This happens because developers often take a top-down approach, treating community engagement as a box to tick for project approval rather than a true partnership. To build a truly fair and sustainable energy system, we need to change how we work. The solution is to adopt open science. This approach can fix these ethical problems by sharing information freely and giving communities the power to shape their own energy future, especially through tools like Open Source Appropriate Technology.

In many energy projects, the way communities are engaged is designed to control opposition, not to build partnerships. This is a deep ethical failure. It often looks like a "tick-box" exercise where developers hold meetings just to satisfy legal rules, without any real intention of using the feedback they receive \citep{ryder2023}. This breeds distrust. We also see this in Community Benefit Agreements (CBAs), which can become simple transactions to buy a community's permission, rather than good-faith efforts to reduce a project's negative impacts \citep{eisenson2023}. This approach fails to deliver what is known as 'energy justice'. True energy justice isn't just about sharing the financial benefits of a project fairly. According to researchers like \citet{sovacool2015} and \citet{miller2014}, it also means that decision-making processes must be fair and inclusive, and that different cultures and values must be respected. The current top-down model fails on both of these points. This problem is made worse by the closed nature of the research that backs these projects. When data is kept secret, communities are left without power and public trust is damaged \citep{alonso2025}.

Open science offers a direct solution. By promoting transparency and collaboration, it can make the energy transition fairer and more effective. It fights the culture of secrecy in research by making data, code, and methods freely available. This allows for others to check the work and builds the public trust we need to tackle climate change. More importantly, open science addresses the core problem of power imbalance. When information is shared openly, communities can participate in decisions in a meaningful way. Access to information allows them to move from the sidelines to the center of the process.

Open Source Appropriate Technology (OSAT) is where this principle comes to life. Instead of imposing a one-size-fits-all product, OSAT provides flexible designs that can be adapted to local needs. For example, an open-source design for solar panel racking can be built with wood or metal depending on local costs, giving local people the power to make their own design choices and lowering project expenses \citep{rana2023}. Another example is the creation of low-cost, open-source lab equipment like a solar-powered ball mill, which makes the tools of science available to everyone \citep{mottaghi2023}. As argued by \citet{pierce2019}, sharing this kind of knowledge and capability is key. OSAT turns communities from passive customers into active builders and owners.

The injustices we see in the energy transition are not accidents; they are the direct result of a closed, top-down way of working. The only way to fix this is to shift to an open model. Open science is not just about better research; it is about changing who holds power. By providing the tools and the framework to share information freely, it allows communities to become active architects of their own energy future, not just bystanders. This is how we can build an energy system that is both sustainable for the planet and just for all of its people.

\bibliographystyle{plainnat}
\bibliography{references}

\end{document} 