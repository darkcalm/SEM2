\documentclass[12pt, letterpaper]{article}
\usepackage[utf8]{inputenc}
\usepackage{amsmath}
\usepackage{amsfonts}
\usepackage{amssymb}
\usepackage{graphicx}
\usepackage[margin=1in]{geometry}
\usepackage{natbib}
\usepackage{hyperref}

\title{Beyond the Bottom Line: Democratizing Power in the Energy Transition}
\author{Gemini}
\date{\today}

\begin{document}

\maketitle

\section*{Introduction}

The global energy transition presents a profound paradox: a project initiated to ensure planetary sustainability that frequently deepens local and social inequities. While framed as a universal good, its dominant, top-down implementation often perpetuates energy injustice. This occurs through an "instrumental" approach to community engagement that prioritizes project acceptance over genuine partnership. A truly just and sustainable transition requires a paradigm shift, moving beyond flawed benefit-sharing schemes to embrace open science and community-led models that build local wealth and democratize power.

\section*{The Illusion of Participation: Deconstructing "Instrumental" Community Engagement}

The prevailing model of community engagement in many energy projects is ethically and functionally bankrupt. It is built on a foundation of instrumentalism, where engagement is not a goal in itself but a means to an end---namely, securing the social license to operate. This approach is characterized by a "tick-box" mentality, where procedural hurdles are cleared without meaningful incorporation of community feedback \citep{ryder2023}. 

This manifests clearly in the typical application of Community Benefit Agreements (CBAs). Rather than serving as tools for genuine partnership and harm mitigation, CBAs are often wielded as transactional instruments to "buy" community support, framing the relationship as a financial one rather than a relational one \citep{eisenson2023}. This practice fundamentally misunderstands the nature of justice. By focusing almost exclusively on distributive justice (financial payouts), it ignores the critical dimensions of procedural justice (who holds power in decision-making) and recognition justice (whose values are respected), thereby reinforcing the very power imbalances a just transition should seek to dismantle \citep{sovacool2015}.

\section*{The Alternative: Building Power Through Community-Led Models and Open Science}

A shift from transactional relationships to transformative partnerships is not only possible, but essential. This requires moving beyond simple compensation and towards a strategy of Community Wealth Building (CWB), which focuses on how energy projects can serve as catalysts for creating democratic, locally-controlled economies \citep{hannon2023}. Instead of asking "How much do we need to pay?", the question becomes "How can this project build lasting local wealth, jobs, and resilience?"

Concrete alternative models already exist. The concept of "community energy twinning" between Global North and South organizations, for instance, fosters collaboration and knowledge sharing rather than top-down extraction \citep{eales2024}. This embodies a different kind of relationship, one built on solidarity and shared goals.

This is where the promise of open science, as articulated by \citet{pierce2019}, becomes a critical enabler. Open science---through open data, open-source hardware, and transparent methodologies---is the toolbox for this new paradigm. It breaks down the information asymmetry that has historically disempowered communities. When citizens have access to the same data and tools as developers, they can become active participants, co-designers, and even owners of their energy future, not merely its stakeholders.

\section*{Conclusion}

The chasm between a truly just energy transition and our current trajectory is defined by the difference between transactional and transformative engagement. To close this gap, a dual commitment is necessary. First, we need a fundamental shift in our ethical frameworks, moving away from instrumentalism and towards justice-centered approaches like Community Wealth Building. Second, we need a commitment to deploying the tools of open science that can make this ethical shift a practical reality. Only by democratizing both knowledge and power can we hope to build an energy future that is not only sustainable for the planet, but also just for its people.

\bibliographystyle{plainnat}
\bibliography{references}

\end{document} 